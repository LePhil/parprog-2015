\documentclass[8pt]{extarticle}
\usepackage{helvet}
\usepackage[T1]{fontenc}
\usepackage[a4paper,landscape,margin=1cm]{geometry}
\usepackage[utf8]{inputenc}
\usepackage[english]{babel}
\usepackage{xcolor}

\usepackage{listings}
\usepackage{tcolorbox}
\tcbuselibrary{breakable}
\tcbuselibrary{skins}
\usepackage{caption}
\usepackage{multicol}

\BeforeBeginEnvironment{lstlisting}{\begin{tcolorbox}[boxsep=-3mm]}
\AfterEndEnvironment{lstlisting}{\end{tcolorbox}}

\lstset{basicstyle=\tiny}

\renewcommand{\familydefault}{\sfdefault}

\let\oldtextbf\textbf
\renewcommand{\textbf}{\tiny\oldtextbf}

\parindent0pt

\begin{document}
\begin{multicols*}{4}
\huge{ParProg 2015}\\\\
\textbf{[Java Thread]}
In Java können Threads mittels Vererbung hergestellt werden. Hierzu gibt es die Basisklasse \textbf{Thread}. Beim ableiten des von der Klasse muss nur \textbf{public void run()} überschrieben werden. Für den Start des Thread muss dann die Methode \textbf{start()} aufgerufen werden. EIN AUFRUF VON \textbf{run()} WÜRDE BLOCKIEREN!
\begin{lstlisting}[language=java]
public class MyApp {
  private static final class MyThread extends Thread {
    public void run() {
      do_something();
    }
  }
  //...
  public static void main(String args[]) {
    (new MyThread()).start();
  }
}
\end{lstlisting}
Des Weiteren gibt es das \textbf{Runnable} Interface, welches ähnlich dem Thread funktioniert. Es von einer Klasse implementiert werden. Auch hier muss die Methode \textbf{public void run()} überschrieben werden. Jedoch muss dann eine Instanz der neuen Klasse der Klasse Thread im Konstruktor übergeben werden.
\begin{lstlisting}[language=java]
public class MyApp {
  private static final class MyRunnable implements Runnable {
    public void run() {
      do_something();
    }
  }
  //...
  public static void main(String args[]) {
    (new Thread(new MyRunnable())).start();
  }
}
\end{lstlisting}
\textbf{Lambdas} sind eine weitere Möglichkeit um einen Thread zu erstellen. Hier wird dem Konstruktor von \textbf{Thread} ein Lambda übergeben.
\begin{lstlisting}[language=java]
public class MyApp {
  public static void main(String args[]) {
    (new Thread(() -> {
      do_something();
    })).start();
  }
}
\end{lstlisting}
Eine vierte Möglichkeit wäre eine \textbf{anonyme innere Klasse}.
\begin{lstlisting}[language=java]
public class MyApp {
  public static void main(String args[]) {
    (new Thread(new Runnable() {
      @Override
      public void run() {
        do_something();
      }
    })).start();
  }
}
\end{lstlisting}
Eine spezielle Art von Threads sind \textbf{Daemon Threads}. Diese Threads verhindern nicht das beenden des Programms. Der \textbf{Garbage Collector} ist ein solcher Thread.
\begin{lstlisting}[language=java]
public class MyApp {
  public static void main(String args[]) {
    // ...
    someThread.setDaemon(true);
    someThread.start();
    // Wait for user input
    System.in.read();
  }
}
\end{lstlisting}
Mit \textbf{Runtime.getRuntime().availableProcessors()} kann man bestimmen wieviele Prozessoren in einem System vorhanden sind. Die maximale Anzahl Threads errechnet sich aus dem verfügbaren Speicher des Systems und der grösse eines Threads. z.B. entsprechen ca. 110'000 Threads auf Windows 64-Bit in etwa 7GB Speicher. Bei Mehr Threads beginnt das System zu swappen und wird langsam.\\
\textbf[synchronized]\\
Für einfachen (ineffizienten) \textbf{gegenseitigen Ausschluss} (mutual exclusion) gibt es das \textbf{synchronized} Schlüsselwort. Es muss vor dem Typ der Funktion angegeben werden. Der Auschluss via synchronized funktioniert mittels eines \textbf{Monitor-Locks} (siehe Monitor Object [POSA2]). Es kann immer nur ein Thread eine synchronized Methode auf einem Objekt ausführen. Alle anderen müssen warten \textbf{selbst wenn sie eine andere synchronized Methode aufrufen}. \textbf{synchronized} kann auch auf Block-level verwendet werden. Die Synchronisierung mittels \textbf{synchronized} ist \textbf{reentrant}, das heisst ein Thread kann mehrere \textbf{synchronized} Methoden verschachtelt aufrufen.
\begin{lstlisting}[language=java]
public class MyApp {
  public synchronized void syncMethOne(){
    dangerous_things();
  }

  public synchronized void syncMethTwo(){
    dangerous_things();
  }

  public void partlySync(){
    // do non dangerous stuff
    synchronized(this) {
      dangerous_things();
    }
  }
}
\end{lstlisting}
Da jedes Objekt ein \textbf{Monitor Lock} besitzt kann auf jedem Objekt ge\textbf{synchronized} werden (also auch auf Member-Objekten). Bedingungen sollten in \textbf{synchronized} Methoden mit \textbf{while} geprüft werden da sonst die gefahr besteht dass man fälschlicher Weisse weiter macht. Warten kann man mit \textbf{wait()} und notifizieren mit \textbf{notify} und \textbf{notifyAll()} wobei \textbf{notify()} \textbf{irgendeinen (!!!) wartenden Thread} weckt. FAUSTREGEL: Bei unterschiedlichen Wartebedingungen mit \textbf{notifyAll()} wecken.\\\\
\textbf{[Semphore]} sind "Zähler". Mit \textbf{acquire()} wird eine "Marke" entfernt, mit \textbf{release()} eine zurückgelegt. \textbf{acquire()} blockiert falls gerade keine "Marke" frei ist. \textbf{Semaphore} können fair sein (\textbf{new Semaphore(n, true)}) und folgen dan dem FIFO-Prinzip.\\\\
\textbf{[Locks \& Conditions]} können verwendet werden um ähnliche Funktionalität wie beim \textbf{Monitor} zu erreichen, mit dem grossen Unterschied, dass hier die Wartebedingungen gezielt "notifiziert" (via \textbf{signal()}) werden können. Jedem \textbf{Lock} können mehrere \textbf{Conditions} zugeordnet sein auf welche Threads via \textbf{await()} warten können.
\begin{lstlisting}[language=java]
public class MyApp {
  // fair lock
  private Lock mon = new ReentrantLock(true);
  private Condition con1 = mon.newCondition();
  private Condition con2 = mon.newCondition();

  public void put(){
    mon.lock()
    try {
      while(canPut == false) {
        con1.await();
      }
      // do put
      con2.signal();
    } finally { mon.unlock(); }
  }

  public void get(){
    mon.lock()
    try {
      while(canGet == false) {
        con2.await();
      }
      // do put
      con1.signal();
    } finally { mon.unlock(); }
  }
}
\end{lstlisting}
\end{multicols*}

\end{document}
