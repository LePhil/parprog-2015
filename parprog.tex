\documentclass[8pt]{extarticle}
\usepackage{helvet}
\usepackage[T1]{fontenc}
\usepackage[a4paper,landscape,margin=1cm]{geometry}
\usepackage[utf8]{inputenc}
\usepackage[english]{babel}
\usepackage{xcolor}

\usepackage{listings}
\usepackage{tcolorbox}
\tcbuselibrary{breakable}
\tcbuselibrary{skins}
\usepackage{caption}
\usepackage{multicol}

\BeforeBeginEnvironment{lstlisting}{\begin{tcolorbox}[boxsep=-3mm]}
\AfterEndEnvironment{lstlisting}{\end{tcolorbox}}

\lstset{basicstyle=\tiny}

\renewcommand{\familydefault}{\sfdefault}

\let\oldtextbf\textbf
\renewcommand{\textbf}{\tiny\oldtextbf}

\parindent0pt

\begin{document}
\begin{multicols*}{5}
\huge{ParProg 2015}\\\\
\textbf{[Java Thread]}
In Java können Threads mittels Vererbung hergestellt werden. Hierzu gibt es die Basisklasse \textbf{Thread}. Beim ableiten des von der Klasse muss nur \textbf{public void run()} überschrieben werden. Für den Start des Thread muss dann die Methode \textbf{start()} aufgerufen werden. EIN AUFRUF VON \textbf{run()} WÜRDE BLOCKIEREN!
\begin{lstlisting}[language=java]
class MyThread extends Thread {
  public void run() {
    do_something();
  }
}
//...
public static void main(String args[]) {
  (new MyThread()).start();
}
\end{lstlisting}
Des Weiteren gibt es das \textbf{Runnable} Interface, welches ähnlich dem Thread funktioniert. Es von einer Klasse implementiert werden. Auch hier muss die Methode \textbf{public void run()} überschrieben werden. Jedoch muss dann eine Instanz der neuen Klasse der Klasse Thread im Konstruktor übergeben werden.
\begin{lstlisting}[language=java]
class MyRunnable implements Runnable {
  public void run() {
    do_something();
  }
}
//...
public static void main(String args[]) {
  (new Thread(new MyRunnable())).start();
}
\end{lstlisting}
\textbf{Lambdas} sind eine weitere Möglichkeit um einen Thread zu erstellen. Hier wird dem Konstruktor von \textbf{Thread} ein Lambda übergeben.
\begin{lstlisting}[language=java]
public static void main(String args[]) {
  (new Thread(() -> {
    do_something();
  })).start();
}
\end{lstlisting}
Eine vierte Möglichkeit wäre eine \textbf{anonyme innere Klasse}.
\begin{lstlisting}[language=java]
public static void main(String args[]) {
  (new Thread(new Runnable() {
    @Override
    public void run() {
      do_something();
    }
  })).start();
}
\end{lstlisting}
Eine spezielle Art von Threads sind \textbf{Daemon Threads}. Diese Threads verhindern nicht das beenden des Programms. Der \textbf{Garbage Collector} ist ein solcher Thread.
\begin{lstlisting}[language=java]
public static void main(String args[]) {
  // ...
  someThread.setDaemon(true);
  someThread.start();
  // Wait for user input
  System.in.read();
}
\end{lstlisting}
Mit \textbf{Runtime.getRuntime().availableProcessors()} kann man bestimmen wieviele Prozessoren in einem System vorhanden sind. Die maximale Anzahl Threads errechnet sich aus dem verfügbaren Speicher des Systems und der grösse eines Threads. z.B. entsprechen ca. 110'000 Threads auf Windows 64-Bit in etwa 7GB Speicher. Bei Mehr Threads beginnt das System zu swappen und wird langsam.\\\\
\textbf{[synchronized]} Für einfachen (ineffizienten) \textbf{gegenseitigen Ausschluss} (mutual exclusion) gibt es das \textbf{synchronized} Schlüsselwort. Es muss vor dem Typ der Funktion angegeben werden. Der Auschluss via synchronized funktioniert mittels eines \textbf{Monitor-Locks} (siehe Monitor Object [POSA2]). Es kann immer nur ein Thread eine synchronized Methode auf einem Objekt ausführen. Alle anderen müssen warten \textbf{selbst wenn sie eine andere synchronized Methode aufrufen}. \textbf{synchronized} kann auch auf Block-level verwendet werden. Die Synchronisierung mittels \textbf{synchronized} ist \textbf{reentrant}, das heisst ein Thread kann mehrere \textbf{synchronized} Methoden verschachtelt aufrufen.
\begin{lstlisting}[language=java]
public class MyClass {
  public synchronized void syncMethOne(){
    dangerous_things();
  }

  public synchronized void syncMethTwo(){
    dangerous_things();
  }

  public void partlySync(){
    // do non dangerous stuff
    synchronized(this) {
      dangerous_things();
    }
  }
}
\end{lstlisting}
Da jedes Objekt ein \textbf{Monitor Lock} besitzt kann auf jedem Objekt ge\textbf{synchronized} werden (also auch auf Member-Objekten). Bedingungen sollten in \textbf{synchronized} Methoden mit \textbf{while} geprüft werden da sonst die gefahr besteht dass man fälschlicher Weisse weiter macht. Warten kann man mit \textbf{wait()} und notifizieren mit \textbf{notify} und \textbf{notifyAll()} wobei \textbf{notify()} \textbf{irgendeinen (!!!) wartenden Thread} weckt. FAUSTREGEL: Bei unterschiedlichen Wartebedingungen mit \textbf{notifyAll()} wecken.\\\\
\textbf{[Semphore]} sind "Zähler". Mit \textbf{acquire()} wird eine "Marke" entfernt, mit \textbf{release()} eine zurückgelegt. \textbf{acquire()} blockiert falls gerade keine "Marke" frei ist. \textbf{Semaphore} können fair sein (\textbf{new Semaphore(n, true)}) und folgen dan dem FIFO-Prinzip.\\\\
\textbf{[Locks \& Conditions]} können verwendet werden um ähnliche Funktionalität wie beim \textbf{Monitor} zu erreichen, mit dem grossen Unterschied, dass hier die Wartebedingungen gezielt "notifiziert" (via \textbf{signal()}) werden können. Jedem \textbf{Lock} können mehrere \textbf{Conditions} zugeordnet sein auf welche Threads via \textbf{await()} warten können.
\begin{lstlisting}[language=java]
public class MyClass {
  // fair lock
  private Lock mon = new ReentrantLock(true);
  // Conditions
  private Condition con1 = mon.newCondition();
  private Condition con2 = mon.newCondition();

  public void put(){
    mon.lock()
    try {
      while(canPut == false) {
        con1.await();
      }
      // do put
      con2.signal();
    } finally { mon.unlock(); }
  }

  public void get(){
    mon.lock()
    try {
      while(canGet == false) {
        con2.await();
      }
      // do put
      con1.signal();
    } finally { mon.unlock(); }
  }
}
\end{lstlisting}
\textbf{[Read-Write-Locks]} erlauben feingranulares Sperren. Da es nicht nötig ist zu sperren wenn alle Threads \textbf{nur lesen} können beliebig viele Threads gleichzeitig ein \textbf{Read-Lock} halten, jedoch \textbf{nur einer} ein \textbf{Write-Lock}. \textbf{Read-Write-Locks} können NICHT geupgraded werden.\\\\
\textbf{[CountdownLatch]} bietet einen "Einweg-Synchronisationspunkt". Bei der Initialisierung muss angegeben werden soll auf wieviele Threads gewartet wird. Threads verwenden \textbf{countDown()} auf dem Latch um den Zähler zu dekrementieren. Mit \textbf{await()} wird gewartet bis der Latch auf 0 ist. Der Latch kann nicht wiederverwendet werden.\\\\
\textbf{[CyclicBarrier]} ist ähnlich wie \textbf{CountdownLatch}. Aber \textbf{CyclicBarrier} kann wiederverwendet werden und wird bei \textbf{await()} dekrementiert. \textbf{getParties()} kann verwendet werden um die Anzahl Teilnahmer der Barriere auszulesen.\\\\
\textbf{[Phaser]} stellt eine verallgemeinerte \textbf{CyclicBarrier} dar. Mit \textbf{arriveAndAwaitAdvance()} wird auf die Freigabe gewartet. Threads können sich mit \textbf{register()} am Phaser an- und mit \textbf{arriveAndDeregister} abmelden.\\
\textbf{[Exchanger]} bietet eine Möglichkeit zwischen zwei Threads Objekte auszutauschen. \textbf{exchange(obj1)} wartet bis der andere Thread auch \textbf{exchange(obj2)} aufgerufen hat.\\\\
\textbf{[Race Conditions]} werden in \textbf{low-level (Data Race)} und \textbf{high-level (Semantic Race)} unterteilt. \textbf{Low-level} Races treten auf wenn unsynchronisiert auf den gleichen Speicher (Variable, Array-Element, ...) zugegriffen wird. \textbf{High-level} Races sind Race-Conditions in der Programmlogik. Sie Treten auf wenn die Critical-Sections nicht ausreichend geschützt sind.\\\\
Auf Synchronisierung kann verzichtet werden wenn entweder \textbf{Unveränderlichkeit} (z.B Java final) gegeben ist oder veränderlich Objekte in breits synchnosierten Objekten eingesperrt sind (\textbf{Confinement}).\\\\
\textbf{[Collections]} aus \textbf{java.util} sind grundsätzlich nicht Threadsafe. Es gibt jedoch Thread-Safe-Collections in \textbf{java.util.concurrent}.\\\\
\textbf{[Deadlocks]} treten auf wenn sich Threads gegenseitig sperren. Folgende Bedingungen müssen eintreten damit es einen Deadlock gibt: \textbf{Geschachtelte Locks} UND \textbf{Zyklische Warteabhängigkeiten} UND \textbf{Gegenseitiger Ausschluss} UND \textbf{Kein Timeout}. Deadlocks lassen sich durch Einführen einer \textbf{linearen Sperrordnung} oder \textbf{grobgranularer Sperrung} lösen.\\\\
\textbf{[Starvation]} tritt ein wenn einem Thread immer wieder die Möglichkeit zu laufen "weggeschnappt" wird. Dies lässt sich durch \textbf{faire} Synchronisationsprimitiven lösen.\\\\
\textbf{[Thread-Pools]} besitzen eine \textbf{Task Queue} in welcher Tasks eingereit und dann von einem freien \textbf{Worker Thread} bearbeitet werden. In Java erzeugt man \textbf{Thread Pools} mit der Factory Klasse \textbf{Executors}. Zur Auswahl stehen \textbf{newFixedThreadPool(nofThreads)}, \textbf{newChachedThreadPool()} (automatische Thread Zahl) und \textbf{newWorkStealingPool()}. Gesondert gibt es noch den \textbf{ForkJoinPool}, welcher es erlaubt rekursive Tasks zu formulieren. \textbf{Thread Pools} haben den Vorteil, dass nicht unnötig viele Threads erzeugt werden. Eine Einschränkung ist, dass Tasks nicht von einander abhängig sein dürfen da sonst eine \textbf{Deadlock}-Gefahr besteht. Der \textbf{ForkJoinPool} setzt im Gegensatz zu den anderen Pools \textbf{Daemon Thread}s ein. Tasks werden mittels des \textbf{Callable} Interfaces implementiert:
\begin{lstlisting}[language=java]
class CalcTask implements Callable<Integer> {
  @Override
  public Integer call() throws Exception {
    int val = 42;
    // long running stuff
    return val;
  }
}

ExecutorService pool =
  Executors.newFixedThreadPool(2);
Future<Integer> fut1, fut2;
// ...
fut1 = pool.submit(new CalcTask());
fut2 = pool.submit(new CalcTask());
// ...
int res1 = fut1.get();
int res2 = fut2.get();
// ...
pool.shutdown();
\end{lstlisting}
\textbf{[Futures]} repräsentieren Zukünftige Ergebnisse (sie sind Proxies im sinne des PROXY Patterns [GoF]) welche mittels \textbf{get()} abgeholt werden. \textbf{get()} blockiert bis das Ergebenis da ist! Wenn eine unbehandelte Exception aufgetreten ist liefert \textbf{get()} diese zurück geschachtelt in einer \textbf{ExecutionException}. Mittels \textbf{cancel()} kann ein Task aus der Warteschlange entfert werden (bricht aber nicht einfach den laufenden Task ab).\\\\
\textbf{[Work Stealing]} ist ein verfahren das in Thread Pools eingesetzt wird bei welchem es eine \textbf{globale FIFO} Queue und für jeden Worker Thread eine \textbf{locale LIFO} Queue gibt. Tasks in der globalen Queue werden (per Default) "vernachlässigt" aber der Work Stealing Pool kann auch auf FIFO umgestellt werden.\\\\
\textbf{[Asynchrone Aufrufe]} erlauben das Auslagern von langen Operationen auf einen anderen Thread oder Pool. Es gibt zwei ansätze: Caller-zentrisch (\textbf{pull}) und Callee-zentrisch (\textbf{push}). Pull setzt auf \textbf{Future} Objekte währen pull auf \textbf{Completion Callback}s setzt. Für Comletion Callbacks eignet sich als Interface zum Beispiel \textbf{java.util.function.Consumer}.
\begin{lstlisting}[language=java]
interface Consumer<T> {
  void accept(T result);
}
// ...
void asyncOp(int in, Consumer<Integer> cb) {
  pool.submit(() -> {
    Integer res = longOperation(in);
    cb.accept(res);
  });
}
// ...
asyncOp(42, res -> {
  System.out.println(res);
});
\end{lstlisting}
\textbf{[Continuations]} bieten in Java 8 eine Möglichkeit eine Folgeoperation an einen Task anzuhängen. Dazu verwendet man \textbf{CompletableFuture<T>}:
\begin{lstlisting}[language=java]
CompletableFuture<Long> fut = 
  CompletableFuture.supplyAsync(() -> longOp());
//...
fut.thenAccept(res -> System.out.println(res));
\end{lstlisting}
Die Continuation läuft auf dem Initiator \textbf{NUR WENN} die Future das Ergebnis schon hat, sonst auf einem beliebigen Worker Thread. Mittels \textbf{allOf()} und \textbf{any()} kann eine Continuation an mehrere Futures gebunden werden:
\begin{lstlisting}[language=java]
// wait for ALL Futures to arrive
CompletableFuture.allOf(fut1, fut2)
  .thenAccept(cont);
// wait for ANY Future to arrive
CompletableFuture.any(fut1, fut2)
  .thenAccept(cont);
\end{lstlisting}
\textbf{[.NET]} Threads "funtionieren" ähnlich wie Java Threads, nur ohne Vererbung sondern mit \textbf{Delegates} und mit dem Unterschied, dass \textbf{uncaught Exceptions} per Default zum \textbf{Programmabbruch} führen. Hier ein Threading-Beispiel mit einem Lambda:
\begin{lstlisting}[language=c++]
Thread myThread = new Thread(() => {
  for(int i = 0; i < 100; ++i)
    {
    Console.WriteLine("Step {0}", i);
    }
});
// ...
myThread.Start();
myThread.Join();
\end{lstlisting}
Im Gegensatz zu Java 8 können in .NET Lambdas auch \textbf{non-final} Variablen aus dem umgebenden Scope (via Referenzen) zugegriffen werden (AUCH SCHREIBEND!!!) was die Chance auf \textbf{low-level} Races erhöht.\\\\
\textbf{[Delegates]} sind vergleichbar mit Funktionszeigern welche ein implizites \textbf{this} besitzen. Sie sind damit leichgewichtiger als Java Interfaces, da sie keine Vererbungshierarchie aufweisen. Lambdas sind im Prinzip eine Abbildung von Delegates.\\\\
\textbf{[Monitor]} ist in .NET mittels des Schlüsselworts \textbf{lock} verfügbar, welches ähnlich wie \textbf{synchronized} vor einen Block gehängt wird. Als Best-Practice hat es sich eingebürgert ein Hilfsobjekt und nicht das eigene Objekt zu sperren. Zum warten und benachrichten werden \textbf{Monitor.Wait(lock)}, \textbf{Monitor.Pulse()} und \textbf{Monitor.PulseAll()} verwendet. Der .NET Monitor ist fair und \textbf{Pulse()} weckt immer den Thread der schon am längsten wartet.
\begin{lstlisting}[language=c++]
class Account {
  // ...
  private object syncObject = new object();

  public void syncOp1() {
    //...
    lock(syncObject) {
      dangerous_things();
      Monitor.PulseAll();
    }
  }
}
\end{lstlisting}
\textbf{[Andere Primitiven]} in .NET sind analog zu Java, ausser \textbf{Locks \& Conditions} welche fehlen und auch die fehelenden Fairness Flags. Dazu kommen \textbf{ReadWriteLockSlim} (upgradable), \textbf{Mutex} (binärer Semaphor) und speziellere. \textbf{Ausser} den Collections in \textbf{System.Collections.Concurrent} sind alle Collection NICHT Threadsafe.\\\\
\textbf{[TPL]} (Task Parallel Library) ist ein \textbf{Work Stealing Thread Pool} mit versciedenen Abstraktionsebenen (Task Parallelism, Data Parallelism, Asynchronous Programming with Continuations). Die \textbf{TPL} erkennt geschachtelte Task selber und es sind keine "speziellen" Vorkehrungen zu treffen. Des Weiteren spawned die \textbf{TPL} selber neue Threads wenn sie merkt das alle Threads blockiert sind. ACHTUNG: \textbf{TPL} Threads sind Background Threads (analog Java Daemon Threads). Via \textbf{ThreadPool.SetMaxThreads()} kann die maximale Anzahl Threads festgelegt werden -> \textbf{Deadlock}-Gefahr
\begin{lstlisting}[language=c++]
Task task1 = Task.Factory.StartNew(() => {
  // do stuff
});

// wait for task to finish
task1.Wait();

// tasks can have return values
Task<int> task2 = Task.Factory.StartNew(() => {
  int res = 42;
  return res;
});

// this blocks until task2 is finished
Console.WriteLine(task2.Result);
\end{lstlisting}
\textbf{[Exceptions]} in Tasks werden seit .NET 4.5 stillschweigend ingoriert und müssen explizit via den Event \textbf{TaskScheduler.UnobservedTaskException} aboniert werden.\\\\
\textbf{[Datenparallelität]} kann mittels Bordmitteln erreicht werden. Voraussetzung ist, dass die Unabhängigkeit der Daten gegeben ist!
\begin{lstlisting}[language=c++]
Parallel.Invoke(
  () => Console.WriteLine("foo"),
  () => Console.WriteLine("bar")
);

Parallel.ForEach(list,
  entry => DoStuff(entry)
);

Parallel.For(0, arr.Length,
  idx => MoreStuff(arr[idx])
);
\end{lstlisting}
\end{multicols*}

\end{document}
